% !Mode:: "TeX:UTF-8"

\addcontentsline{toc}{chapter}{致\quad 谢}%添加到目录中
\chapter*{致\quad 谢}
时光荏苒,岁月如梭,不知不觉中我的博士学习生涯即将结束。四年的博士研究生生活是紧张而又充实的,给我留下了许多美好的回忆,追忆这四年的求学时光,心中不经涌现出无限的怅惘和眷念。借此机会,我想向所有关心、支持和帮助过我的人表达我的感激之情。

首先,我要由衷地感谢我的导师滕召胜教授。本论文的全部研究工作也是在恩师地悉心指导下完成的,从论文的选题、研究工作的开展以及论文的撰写和修改,倾注了滕老师无数的汗水和心血。我于~2005~年参与湖南大学大学生创新训练项目(Students Innovation Training Program,SIT)时与滕老师相识,从那时起,滕老师就以渊博的学识、严谨的作风、豁达的心态和风趣幽默的教学风格给我留下了极其深刻的印象。后承蒙滕老师不弃,将我收入门下,能作为滕门的弟子我倍感荣幸。从本科到硕士再到博士,在我求学中的每个阶段,滕老师都以勤奋忘我的工作作风和勇于创新的探索精神在无形中激励并感染着我。从第一次设计电路板、第一次调试程序、第一次带领项目团队以及第一次参与材料撰写,滕老师给我锻炼机会的同时也给予了我无私的指导。经师易遇,人师难遭!在我遇到困难和挫折的时候,滕老师总是鼓励我,让我能够勇敢乐观、迎难而上。在我小有收获,开始沾沾自喜的时候,滕老师又总是叮嘱我,要沉下来,要往前看,要有大局观。所有的这些都让我铭记于心并终身受益,我惟有在今后的工作中更加努力以报答我的恩师!

衷心感谢唐求老师在论文工作阶段的悉心指导和热情帮助!感谢高云鹏、温和、林海军和王永等师兄在生活和科研中的支持和帮助!感谢李雅鑫、宋俊皓、李菲、何宝、段辉江、何俊杰、吴言、曹振、唐夕晴、孙彪、胡清、钟浩文、桑江艳等师弟师妹的关心和帮助!

感谢湖南大学电气与信息工程学院对我的培养,是她给予我学习与工作平台,才让我有了今日的成长和进步。

感谢我的老教导员李安福在部队时给予我们一家的帮助和照顾,让我能够坚持自己的理想而不轻言放弃!

感谢我美丽贤惠的妻子刘婷,在我的整个博士求学过程中,她独自一人、无怨无悔地承当起繁重的家务和照顾孩子的重任,特别是在二宝出生后,依旧不让我分心学业,让我得以全身心投入到学习和科研当中去,我所有取得的成果都离不开妻子在背后的默默奉献。感谢我聪明伶俐的大儿子李昊霆,每次在我苦于科研毫无头绪而愁眉不展的时候,他总是能给我带来欢乐。感谢我天真可爱的小儿子李昊博,你的到来给我带来了无穷的前进动力。由于学习和科研的繁重,只有很少的时间能够陪伴他们,对此我深感愧疚,但他们总是能够宽容我、理解我并无条件地支持我,伴着我一路微笑前行,他们的宽容、理解和支持是我能够克服重重困难、锐意进取的重要源泉。

感谢我的父母和岳父母,是他们一如既往的无私支持和帮助,让我得以顺利完成博士学业。尤其要感谢我的岳父给予我人生道路上的指引,他宽广的心胸和精益求精的工作态度一直是我学习的榜样。再次感谢所有关心和帮助过我的人!

\rightline{李建闽}

\rightline{二零一八年九月 于湖南大学}





